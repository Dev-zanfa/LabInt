\documentclass{article}
\usepackage{listings}
\usepackage[utf8]{inputenc}
\usepackage{graphicx}
\usepackage{subfiles}
\usepackage{wrapfig}
\usepackage{geometry}
\usepackage[italian]{babel}
\usepackage[export]{adjustbox}
\usepackage[font=scriptsize,labelformat=empty,labelsep=none]{caption}
\usepackage{titlesec}
\usepackage{amsmath}
\usepackage[titletoc]{appendix}

\setcounter{section}{-1}




\titlespacing*{\section}
{0pt}{3ex plus 1ex minus .2ex}{3ex plus .2ex}
\titlespacing*{\subsection}
{0pt}{3ex plus 1ex minus .2ex}{3ex plus .2ex}
\graphicspath{ {./images/} }
\geometry{a4paper, top=2cm, bottom=2cm, left=2cm, right=2cm, heightrounded, bindingoffset=5mm}

\title{\Huge LABORATORIO DI INTERNET 
    \\ Report 2: Analisi di velocita' \\ tramite ping}
\author{Gruppo 21}
\date{Marzo 2021}

\begin{document}
    \maketitle

    \begin{center}
        \includegraphics[scale=0.75]{logo_poli.png}

        \vspace{20mm}

        Diego Zanfardino s256536, \\
        Fabio Trovero s258574, \\
        Lorenzo Ferro s260878

        \vspace{10mm}
        prof. Mellia Marco
    \end{center}

    \pagebreak

    % ========
    % ========

    \section{Introduzione}

    In questo laboratorio analizziamo come poter stimare la velocita' di trasmissione a
    livello fisico tramite misure di RTT, esaminando 4 diverse possibili configurazioni. \\
    Per rendere i grafici piu' accurati possibile e' stato considerato l'effetto dovuto 
    alla frammentazione; quando la dimensione del pacchetto supera la MTU (1500 Byte) 
    calcoliamo il numero di pacchetti risultanti con la seguente formula:
    
    \begin{equation}
        \centering
        D(x) = (x + 8) + (20 + 38)*(1 + floor(\frac{x+8-1}{1480}))
        % D -> dati a livello fisico = S (dimensione a livello applicazione) + 8 + 20 + 38
        % quando l'intestazione supera i 1500 si devono mandare 2 pacchetti al posto di 1
        % quando a livello 3 non posso aggiungere 20 allora creo 2 pacchetti al posto di 1
    \end{equation}
    

    % ========
    % ========


    \section{Configurazione utilizzata punto 1}
    
    \subfile{sections/conf1.tex}

    \subsection{Risultati}
    
    \subfile{sections/ris1.tex}
    
    % ========
    % ========

    \section{Configurazione utilizzata punto 2}

    \subfile{sections/conf2.tex}

    \subsection{Risultati}

    \subfile{sections/ris2.tex}

    % ========
    % ========  

    \section{Configurazione utilizzata punto 3}

    \subfile{sections/conf3.tex}

    \subsection{Risultati}

    \subfile{sections/ris3.tex}

    % ========
    % ========  

    \section{Configurazione utilizzata punto 4}

    \subfile{sections/conf4.tex}

    \subsection{Risultati}

    \subfile{sections/ris4.tex}

    % ========
    % ========  

    \subfile{sections/appendici.tex}

\end{document}