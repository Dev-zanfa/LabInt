\documentclass{article}
\usepackage{listings}
\usepackage[utf8]{inputenc}
\usepackage{graphicx}
\usepackage{subfiles}
\usepackage{wrapfig}
\usepackage{geometry}
\usepackage[italian]{babel}
\usepackage[export]{adjustbox}
\usepackage[font=scriptsize]{caption}
\usepackage{titlesec}
\usepackage{amsmath}
\usepackage[titletoc]{appendix}

\setcounter{section}{-1}




\titlespacing*{\section}
{0pt}{3ex plus 1ex minus .2ex}{3ex plus .2ex}
\titlespacing*{\subsection}
{0pt}{3ex plus 1ex minus .2ex}{3ex plus .2ex}
\graphicspath{ {./images/} }
\geometry{a4paper, top=2cm, bottom=2cm, left=2cm, right=2cm, heightrounded, bindingoffset=5mm}

\title{\Huge LABORATORIO DI INTERNET 
    \\ Report 3: Test performance con\\ off the shell hardware}
\author{Gruppo 21}
\date{Maggio 2021}

\begin{document}
    \maketitle

    \begin{center}
        \includegraphics[scale=0.1]{polito_logo_2021_blu.jpg}

        \vspace{20mm}

        Diego Zanfardino s256536, \\
        Fabio Trovero s258574, \\
        Lorenzo Ferro s260878

        \vspace{10mm}
        prof. Mellia Marco
    \end{center}

    \pagebreak

    % ========
    % ========

    \section{Introduzione}
    In questo laboratorio si cerca di predirre il goodput di un modello di rete semplificato con
    l'utilizzo del comando \textit{iperf3}. Per valutare l'effettiva velocità di connessione tra due 
    host \textit{iperf} utilizza: 
    \begin{equation}
        \centering
        goodput = \frac{ dati\:utili\:(a\:livello\:7)}{\Delta T } = \frac{\eta}{\Delta T } 
    \end{equation}
    Noi siamo invece in grado di predirre il risultato dei test calcolando l'efficienza
    attesa della rete, e successivamente la velocità massima come segue:
    \begin{equation}
        \centering
        V_{MAX} = \eta \cdot C = \frac{S}{S+header_{TCP|UDP}+header_{IP}+header_{ETH}} \cdot C
    \end{equation}
    Dove \textit{C} è la capacità del canale e \textit{S} sono i dati utili 
    scambiati a livello 7. 


    \section{Scenario controllato: lan locale}
    
    \subfile{sections/conf1.tex}

    \subsection{Risultati}
    
    \subfile{sections/ris1.tex}
    
    % ========
    % ========

    \section{Scenario internet}

    \subfile{sections/conf2.tex}

    \subsection{Risultati}

    \subfile{sections/ris2.tex}

    % ========
    % ========  
    \pagebreak
    \subfile{sections/appendici.tex}
    
    \end{document}
    \section{Configurazione utilizzata punto 3}

    \subfile{sections/conf3.tex}

    \subsection{Risultati}

    \subfile{sections/ris3.tex}

    % ========
    % ========  

    \section{Configurazione utilizzata punto 4}

    \subfile{sections/conf4.tex}

    \subsection{Risultati}

    \subfile{sections/ris4.tex}

    % ========
    % ========  
    

